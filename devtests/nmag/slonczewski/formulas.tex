\documentclass[11pt]{article}

\usepackage[colorlinks=true, linkcolor=blue, citecolor=red, urlcolor=red]{hyperref}

\newcommand{\vv}[1]{\vec{#1}}
\newcommand{\vvM}{\vv{M}}
\newcommand{\vvm}{\vv{m}}
\newcommand{\vvH}{\vv{H}_{\mathrm{eff}}}
\newcommand{\vvfix}{\vv{p}}

\newcommand{\Msat}{M_{\mathrm{s}}}

\newcommand{\derivp}[2]{\partial_{#2} #1}

\begin{document}
\section{Implementation of Slonczewski in Nmag}

\subsection{The original equation}
Here we write down the equation as it appears in the original paper
by Slonczewski [JMMM 159 (1996) L1-L7].
First, the quantity $g$ is defined:
$$
g = \left[ -4 + (1 + P)^3 (3 + \vv{s}_1 \cdot \vv{s}_2)/ 4 P^{3/2} \right]^{-1}
$$
$\vv{s}_1$ and $\vv{s}_2$ are two unit vectors.
$\vv{s}_1$ is the direction of the magnetisation in the pinned layer,
$\vv{s}_2$ is the direction of the magnetisation in the free layer.
$P$ is the spin polarisation (of the fixed layer, I guess???).
The LLG equation reads:
$$
\derivp{\vv{S}_2}{t} = 
  \vv{s}_2 \times 
    \left( \gamma H_{\mathrm{u}} \, (\vv{c} \cdot \vv{S}_2) \, \vv{c}
          - \alpha \, \derivp{\vv{S}_2}{t}
          + e^{-1} I_{\mathrm{e}} g \, \vv{s}_1 \times \vv{s}_2 \right)
$$

\subsection{LLG in terms of magnetisation}
The equation above is formulated in terms of spins.
We need an equation which describes the evolution of the magnetisation.
We then follow the procedure described:
\href{http://wpage.unina.it/mdaquino/PhD\_thesis/main/node47.html}{here}.
$$
\derivp{\vvm}{t} =
   -\gamma \, \vvm \times \vvH 
  + \alpha \, \vvm \times \derivp{\vvm}{t}
  + \frac{g_{\mathrm{e}} \mu_{\mathrm{B}} I_{\mathrm{e}} g}{ed\Msat} \vvm \times (\vvm \times \vv{s}_1)
$$
This is basically Eq. (2.87) from the link above.

\subsection{What we actually implement}
We define the following factor:
$$
C(\vvm) = 
  \gamma \,
  \frac{I_{\mathrm{e}} \, \hbar}{\mu_0 \Msat |e| d} \,
  \frac{a(P)}{(3 + \vvm \cdot \vvfix) - 4 a(P)},
\quad
a(P) = 4 \, \left( \frac{\sqrt{P}}{1 + P} \right)^3,
$$
where $J_{\mathrm{e}}$ is the applied current density, $P$ the polarisation,
$d$ the thickness of the free layer, $\vvfix$ the direction of the
magnetisation in the pinned layer (a unit vector).
%
And, $\gamma$ is the gyromagnetic ratio,
$\Msat$ is the saturation magnetisation,
$\vvm = \vvM/\Msat$ the normalised magnetisation, $\mu_0$ the permeability
of free space, $e$ the electric charge of the electron,
$\hbar$ the reduced Plank constant.

The Landau-Lifshitz equation with the Slonczewski spin torque can then be written as:
$$
\derivp{\vvm}{t} = 
  -\gamma \, \vvm \times \vvH
  + \alpha \, \vvm \times \derivp{\vvm}{t}
  + C(\vvm) \, \vvm \times (\vvm \times \vvfix),
$$
where $\alpha$ is the damping constant.
In order to numerically integrate it, we reduce it to the following:
$$
\begin{array}{ccc}
\derivp{\vvm}{t} & = & 
  -\gamma' \, \vvm \times \vvH
  - \alpha \gamma' \, \vvm \times (\vvm \times \vvH) - {} \\
& & {} - \alpha C'(\vvm) \, \vvm \times \vvfix
  + C'(\vvm) \, \vvm \times (\vvm \times \vvfix),
\end{array}
$$
which is obtained by substituting the former into itself and proceeding
with some trivial algebraic manipulations.
$\gamma'$ and $C'(\vvm)$ are defined as
\begin{eqnarray}
\gamma' = \frac{\gamma}{1 + \alpha^2}, \nonumber \\
C'(\vvm) = \frac{C(\vvm)}{1 + \alpha^2}. \nonumber
\end{eqnarray}
\end{document}

